\subsection{Estableciendo una buena semilla}

Luego de analizar la convergencia de los métodos sobre las funciones $f(x)$ y $e(x)$ y estudiar los distintos criterios de parada en función de la magnitud del $\alpha$ en cuestión, procedimos a ver, entender y analizar el problema de la semilla inicial para estos métodos y cómo utilizar el método de bisección como auxiliar para hallar una $buena$ semilla.

Bisección en particular para estas dos funciones es un gran método en cuanto a convergencia, dado que con dos semillas cualesquiera (de distinto signo) bisección converge. El problema recae en su velocidad, y al converger linealmente su eficiencia se ve eclipsada por Newton-Raphson por ejemplo, de convergencia cuadrática. Ahora bien, Newton por hipótesis necesita una semilla $cercana$ a la raiz que queremos hallar o podría diverger...

¿Podemos combinar estos métodos para así encontrar una mejor solución al problema?

Bueno, creemos que si, que utilizando bisección para encontrar semillas tanto para Newton como para Secante puede mejorar notablemente el desempeño de estos métodos, y nuestra hipótesis es que con 4-5 iteraciones de bisección alcanzaría para encontrar una "buena" semilla sin que bisección haga todo el trabajo.

La "lentitud" de bisección recae en los últimos pasos. Este método trabaja siempre sobre un intervalo que en cada paso parte a la mitad y elige una de esas mitades para continuar. El problema es que si el intervalo es chico y el criterio de parada es fuerte, es decir queremos precisión, cada iteracion nos acerca poco al x* deseado. Pero en un principio el intervalo suele ser muy grande, es mucho más lo que se descarta y nos acercamos bastante al valor que queremos hallar. Esta particularidad es la que nos hace pensar que 4 o 5 iteraciones de bisección rápidamente nos puede encontrar una/s semilla/s que Newton o Secante pueden aprovechar.

Para esta etapa de experimentación la idea fue fijar para cada función el criterio de parada que creemos mejor se comporta y luego para distintos valores de $\alpha$ (de distintas magnitudes) determinar cuántas iteraciones de bisección son necesarias para despertar una mejor performance en los métodos que analizamos.
