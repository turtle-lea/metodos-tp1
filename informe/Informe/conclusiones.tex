

En cuanto al problema de la semilla inicial y el número de iteraciones de bisección para el hallazgo de una buena semilla, vimos que combinar bisección con otro método es una buena solución que hace más eficiente el resultado. Definir cuántas iteraciones hacer es más delicado y en muchos casos depende del $\alpha$ en cuestión. 

Llegamos a la conclusión de que sobre $f(x)$, independientemente de cual sea el valor de $\alpha$ unas pocas iteraciones de bisección bastan y como planteamos en la hipótesis creemos que 4-5-6 son suficientes.

Sobre $e(x)$ en cambio el valor de $\alpha$ influye mucho más por lo que una cantidad de iteraciones variable puede ser la solución. Concluimos que con 5 a 9 iteraciones de bisección basta para alphas más o menos cercanos a 1 y aumentar la cantidad de iteraciones para valores de $\alpha$ más chicos o mucho más grandes, con valores entre las 15 y 25 iteraciones.
