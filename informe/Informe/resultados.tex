% % \begin{figure}
% % 	\centering
% 	\begin{minipage}{.5\textwidth}
% 		\centering
% 		\includegraphics[keepaspectratio]{Imagenes/exp1/biseccion_f.pdf}
% 		\captionof{figure}{A figure}
% 		\label{fig:test1}
% 	\end{minipage}%
% 	\begin{minipage}{.5\textwidth}
% 		\centering
% 		\includegraphics[keepaspectratio]{Imagenes/exp1/biseccion_f.pdf}
% 		\captionof{figure}{Another figure}
% 		\label{fig:test2}
% 	\end{minipage}
% \end{figure}

\subsection{Primeras experimentaciones}

\begin{figure}[!h]
	\begin{center}
		  \includegraphics[keepaspectratio]{../Imagenes/exp1/biseccion_f.pdf}
		  \caption{Bisección\_f para $\alpha=4.0, \epsilon = 10^{-6}, criterio = 1\  con \ max\_iter=20$}
		  \label{fig:contra1}
	\end{center}
\end{figure}
\FloatBarrier
~

\begin{figure}[!h]
	\begin{center}
		  \includegraphics[keepaspectratio]{../Imagenes/exp1/biseccion_e.pdf}
		  \caption{Bisección\_e para $\alpha=6.0, \epsilon = 10^{-6}, criterio = 3$}
		  \label{fig:contra1}
	\end{center}
\end{figure}
\FloatBarrier
Los resultados tanto para $biseccion\_f$ como para $biseccion\_e$ fueron los esperados. Los métodos convergen al resultado para cualquier valor de $\alpha$ (Fueron evaluados valores entre 0 y 10000 aproximadamente,
con bastante granularidad). Para realizar los gráficos fueron elegidos los valores de $\alpha$ 4.0 para $f$ y 6.0 para $e$ únicamente debido a que permiten utilizar una escala que nos deja apreciar los aspectos
relevantes del análisis. Los puntos rojos corresponden al valor de $x$ en cada iteración del método. Con el correr de las iteradas, los puntos van amontonándose alrededor de la raíz teórica de la función.

Un aspecto llamativo de los resultados obtenidos en esta etapa es el hecho de que 
a medida que los valores de $\alpha$ crecen, bisección\_e requiere de una mayor cantidad de iteraciones (en comparación con $f$) para obtener un valor cercano a la raíz de la función. Por ejemplo,
tomando 40 iteraciones en cada experimento y variando el valor de alfa, obtuvimos que: 

Para $\alpha = 0.1$, $f(x\_40) \approx 1*10^{-13}$ y $e(x\_40) \approx 1*10^{-13}$. 

Para $\alpha = 7.0$ : 
$f(x\_40) \approx 1*10^{-11}$ y $e(x\_40) \approx 1*10^{-10}$. 

Y ya para $\alpha = 10000.0$: $f(x\_40) \approx 1*10^{-5}$ y $e(x\_40) \approx 0.01$

Notemos que estamos utilizando el criterio (3) para analizar la calidad de la solución obtenida.

Por lo general biseccion\_e se comprta peor que biseccion\_f (Terminar de pensar por qué)

Contrariamente a lo que suponíamos, al analizar newton\_f \ descubrimos que, luego de fijar $\alpha$ (distinto de cero), el método converge para cualquier semilla inicial que tomemos (también distinta de 0).
Pensemos un poco por qué sucede esto: 
Anteriormente ya demostramos que si $seed\_inicial > \sqrt{\alpha}$ entonces el método converge (ejercicio 3 de la práctica). 
Sin embargo, ¿ qué sucede cuando pertenece al itervalo $(0,\sqrt{\alpha})$. $x_1$ es el valor 
obtenido a partir de la interesección entre la recta tangente a $f(x)$ en $(x_0,f(x_0)$ y el eje de abscisas. Experimentalmente parecería ser que el valor de $x_1$ obtenido luego de la primera 
iteración es mayor a $\sqrt{\alpha}$. Este hecho puede explicarse en función la pendiente de la recta tangente en la semilla inicial. Como la misma tiene un valor peque\~no en módulo pero positivo
(menor a $2*\sqrt{\alpha}$), no es ilógico pensar que cortará al eje de abscisas en un valor de $x$ mayor a  $\sqrt{\alpha}$. El siguiente gráfico intenta representar esta idea:  

\begin{figure}[!h]
	\begin{center}
		  \includegraphics[keepaspectratio]{../Imagenes/exp1/recta_tangente.pdf}
		  \caption{Recta tangente a la curva con $x_0=0.25$ y $\alpha=5.0$}
		  \label{fig:contra1}
	\end{center}
\end{figure}
\FloatBarrier

Nótese que en los casos en donde $x_0$ comienza con valores peque\~nos en módulo obtenemos a partir de la primera iteración un $x\_1 >> \sqrt{\alpha}$. A medida que nos fuimos
acercando por izquierda, para valores en donde $x_{0} \approx \sqrt{\alpha}$,
experimentalmente también obtuvimos que $x\_1 > \sqrt{\alpha}$, lo cual explica el hecho de que newton\_f converja para cualquier semilla inicial.

En el siguiente gráfico mostramos un ejemplo de convergencia de newton\_f. Observemos que, a diferencia de bisección, el método converge luego de unas pocas iteraciones: ya no se 
produceese amontonamiento de puntos alrededor de la raíz teórica, a pesar de usar un epsilon 3 órdenes de magnitud más peque\~no 

\begin{figure}[!h]
	\begin{center}
		  \includegraphics[keepaspectratio]{../Imagenes/exp1/newton_f.pdf}
		  \caption{newton\_f para $\alpha=5.25, \epsilon=10^{-8}, \ criterio = 5, \ seed = 1.0$}
		  \label{fig:contra1}
	\end{center}
\end{figure}
\FloatBarrier

En esta figura podemos apreciar por primera vez el comportamiento de una sucesión de punto fijo: el valor $c$ tal que $g(c)=c$ (donde $g(x) = x - \displaystyle \frac{f(x)}{f'(x)}$) es al mismo tiempo
raíz de la función $f$