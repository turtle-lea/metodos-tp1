\rule{\linewidth}{0.5mm}

\vspace{2cm}

Integrantes:
\begin{itemize}
	\item Castro, Dami\'an L.U.: 326/11  \verb+ltdicai@gmail.com+
	\item Matayoshi, Leandro L.U.: 79/11 \verb+leandro.matayoshi@gmail.com+
	\item Szyrej, Alexander L.U.: 642/11   \verb+alexanderszyrej@gmail.com+
	
\end{itemize}

\vspace{2cm}

\begin{abstract}
  El siguiente trabajo práctico tiene como objetivo mostrar una posible forma de encontrar encontrar la raíz cuadrada de un número $\alpha \geq 0$
  utilizando métodos iterativos que convergen a las raíces de funciones reales. Bajo este marco implementamos los algoritmos de Bisección, Newton-Raphson
  y Secante y luego realizamos distintas experimentaciones focalizándonos en distintos aspectos: convergencia o no a la solución buscada, criterios de
  para utilizados, tiempo consumido por cada uno, etc. Finalmente contrastamos los resultados empíricos con lo que era esperable desde el punto de vista
  teórico obteniendo resultados satisfactorios. De esta manera pudimos comprobar que los 3 métodos son útiles en la práctica ya que pudieron
  ser aplicados en una instancia concreta de la vida real: encontrar la raíz cuadrada de un número. Esto puede ser particularmente útil para aplicaciones
  que calculen constantemente normas vectoriales.
\end{abstract}

\vspace{2cm}

Palabras Clave:
\begin{itemize}
	\item Ra\'ices Reales
	\item Newton-Raphson
	\item Convergencia de m\'etodos iterativos
\end{itemize}

