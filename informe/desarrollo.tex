Como habíamos mencionado anteriormente, el objetivo del trabajo es encontrar las raíces de $f(x)=x^{2} - \alpha$ y $e(x) = 1/x^{2} - \alpha$.
Para resolver el problema, implementamos para ambas funciones los cuatro métodos iterativos vistos en clase : Bisección, Newton, Regula Falsi y Secante.
La explicación y la base teórica de los mismos puede ser encontrada fácilmente en distintos libros de métodos numéricos, como
por ejemplo: "Numerical Analysis, Burden \& Faires". 

\subsection{Estructura del código}

Antes de comenzar a escribir código decidimos realizar una breve etapa de diseño, en donde nos propusimos encapsular algunos comportamientos para poder escribir un programa
más prolijo y legible.

Lo primero que hicimos fue diseñar una clase llamada $Funciones$. La idea es la siguiente: sea $h$ la función que estamos analizando (podría ser tanto $f$ como $e$).
Como los métodos requieren evaluar $h$ para calcular el valor de la sucesión en los términos siguientes (y algunos $h\_derivada$), sería bueno encapsular estos cálculos
con un comportamiento similar al de una ''caja negra'' (la famosa caja negra). Esto es: en lugar de efectuar la operación en el scope de la función que ejecuta el método,
invocamos a $Funciones.h(x)$ o $Funciones.h\_derivada(x)$, desligándonos de la forma en que estas están implementadas.

Esto es particularmente notorio y beneficioso, por ejemplo, para el código de Newton\_e (i.e: método de Newton para la función $e$. De ahora en más utilizarmos esta notación
para referirnos a los distintos algoritmos):

$\displaystyle e\_deriv = \frac{-2}{x^{3}} \Rightarrow x_{n+1} = x_{n} - \frac{\frac{1}{x_{n}^2}-\alpha}{\frac{-2}{x_{n}^{3}}} \Rightarrow 
x_{n+1} = x_{n} + \frac{(x_{n} - \alpha x_{n}^{3})}{2} $ 

Newton\_e encapsula todos esos cálculos mediante las operaciones:

~

$x_{n+1} = x_{n} - \frac{Funciones.e(x_{n})}{Funciones.e\_deriv(x_{n})}$

~

Luego, por razones muy similares a la anterior, decidimos abstarer el funcionamiento de los criterios de parada, mediante una clase llamada $Criterios$. La decisión de
''cuándo parar'' es independiente al algoritmo que está siendo ejecutado. Además, eventualmente queremos evaluar el comportamiento de un mismo método con distintos 
criterios. Tomemos como ejemplo Bisección. Si hubiéramos implementado esa parte dentro del código del método, entonces habríamos tenido algo de este estilo:

~

\begin{algorithmic}
\Function{Biseccion}{seeds $positivo,negativo$}
	\State \ldots
	\While{$!(max\_iteraciones < i \lor |medio_i < medio_{i-1}| < \epsilon| \lor \frac{|medio_i < medio_{i-1}|}{|medio_{i-1}|} < \epsilon \ldots)}$ 
		\State $\displaystyle medio_{i-1} = medio$
		\State $\displaystyle medio_{i} = \frac{positivo_{i}+negativo_{i}}{2}$
		\State \ldots
	\EndWhile
\EndFunction
\end{algorithmic}

~

Por el contrario, utilizando la case tenemos:

\begin{algorithmic}
	\While{$!criterios.parar(parameters)$}
		\State \ldots
	\EndWhile
\end{algorithmic}

~

La existencia de class Criterios permite agrupar todo el código que esté relacionado con ellos en un solo bloque, con lo cual ahorramos tener que escribirlos
en cada una de las guardas del while de los distintos algoritmos. Al mismo tiempo resalta el aspecto independiente mencionado más arriba: Los criterios de parada son 
independientes a los métodos que los utilizan.


\subsection{Inicios de la experimentación}

\subsubsection{Primeras experimentaciones. Analizando la convergencia de los métodos}

No surgieron inconvenientes demasiado importantes a la hora de pensar y escribir el código, por lo que procedimos a iniciar la experimentación.
Comenzamos analizando el comportamiento de biseccion\_e y biseccion\_f, ya que asumimos que no tendrían mayores dificultades. Para ello, desarrollamos un método llamado
$semilla\_biseccion\_h$, donde($h = f \ o \ e$), que devuelve 2 valores: $pos$ y $neg$ tales que $h(pos)\geq0$ y $h(neg)$ < 0. La ventaja fundamental de bisección
se basa en los reducidos requisitos iniciales que exige para garantizar convergencia (simplemente encontrar 2 semillas tales que al evaluarlas en $h$ difieran en signo).

Los resultados tanto para $biseccion\_f$ como para $biseccion\_e$ fueron los esperados. Si bien ambos métodos convergieron a las respectivas raíces, un aspecto llamativo es que
a medida que los valores de $\alpha$ crecen, bisección\_e requiere de una mayor cantidad de iteraciones para obtener un valor cercano a la raíz de la función.








