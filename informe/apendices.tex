\subsection{Apéndice A: Enunciado}

\parindent = 0 pt
\parskip = 11 pt

El objetivo del trabajo pr\'actico consiste en implementar un programa que permita calcular, dado $\alpha \in
\mathbb{R}$, $1/\sqrt{\alpha}$. Para ello, se deber\'a considerar las funciones $f(x)$ y $e(x)$ definidas
anteriormente, distintos m\'etodos vistos en clase que permitan resolver el problema planteado y realizar un an\'alisis
completo del comportamiento de los mismos. 

Los requisitos m\'inimos a cumplir son los siguientes:

\begin{itemize}
\item Implementar el m\'etodo de Newton para la funci\'on $f(x)$. Incluir en el informe la demostraci\'on de
convergencia (Ejercicio 3, Pr\'actica 1). Para la funci\'on $e(x)$, implementar al menos dos m\'etodos (uno de los
cuales debe ser el de Newton).   
\item Para cada m\'etodo, estudiar experimentalmente la convergencia, tiempo de ejecuci\'on, cantidad de iteraciones,
criterios de parada, precisi\'on en el resultado, y cualquier otro par\'ametro que considere necesario evaluar. Realizar experimentos
computacionales considerando un rango amplio de valores posibles para $\alpha$ y distintos puntos iniciales
para los m\'etodos. Analizar y justificar detalladamente los resultados obtenidos.
\item Una vez fijados los mejores par\'ametros para cada m\'etodo, realizar una comparaci\'on entre las tres formas
alternativas de resolver el problema (Newton para $f(x)$, y Newton m\'as el otro m\'etodo para $e(x)$) en t\'erminos de
tiempo de ejecuci\'on, precisi\'on en la soluci\'on, cantidad de iteraciones, etc. Determinar experimentalmente que
variante seleccionar\'ia para su utilizaci\'on en la pr\'actica.
\end{itemize}

\subsection{Apéndice B: Código}

\subsection{Apéndice C: Demostración}

Sea $f(x) = x^2 -a$ una funci\'on a la cual se le quiere encontrar raices.Para ello se propone una sucesi\'on de punto fijo $x_{n+1} = g(x_n) = \frac{1}{2}\left(x_n+ \frac{a}{x_n}\right)$. Veamos si efectivamente $g(x)$ corresponde con el m\'etodo de Newton.

Sea $f(x) = x^2 - a$ $\Rightarrow$ $f'(x) = 2x$. Por lo tanto, la sucesi\'on del m\'etodo de Newton es la siguiente: $g_{2}(x_n) = x_n - \frac{(x_{n}^2 -a)}{2x_n}$. 

Queremos ver que $g_2(x_n)=g(x_n)$ 

$g_{2}(x_n) = x_n - \frac{(x_{n}^2 -a)}{2x_n}$ = $\frac{1}{2}\left(2x_n - \frac{x_{n}^2 - a}{x_n}\right) = \frac{1}{2}\left(\frac{2x_{n}^2 -x_{n}^2 + a}{x_n}\right) = \frac{1}{2}\left(\frac{x_{n}^2 + a}{x_n}\right) = \frac{1}{2}\left(x_{n} + \frac{a}{x_n}\right) = g(x_n)$

Luego hay que probar que si $x_0 > \sqrt{a}$ entonces $x_{n+1} < x_n$ para $n \geq 0$. Vamos a hacer esto por inducci\'on completa.

Sea $P(n):=$ $x_0 > \sqrt{a} \Rightarrow x_{n+1} < x_{n}$
\begin{itemize}
	\item Caso base: $P(0) = x_0 > \sqrt{a} \Rightarrow x_1 < x_0$
	\item Caso inductivo: $((\forall n\geq 0)$ $P(n)) \Rightarrow P(n+1)$
\end{itemize}

Caso base:
\begin{center}
	$\displaystyle{x_1 < x_0 \equiv g(x_0) < x_0 \equiv \newline}
	\frac{1}{2}\left(x_0 + \frac{a}{x_0}\right) < x_0 \equiv \left(x_0 + \frac{a}{x_0}\right) < 2x_0 \equiv \newline
	\frac{a}{x_0} < x_0 \equiv a < x_{0}^2$ \footnote{no se invierte el signo porque $x_0 > \sqrt{a} \geq 0$}
	$\Rightarrow \sqrt{a} \leq |x_0| \equiv \sqrt{a} \leq x_0 \equiv P(0)$
\end{center}

Caso inductivo:
 
\begin{center}
	$\displaystyle{x_{n+2} < x_{n+1} \equiv g(x_{n+1}) < x_{n+1} \equiv \newline}
	\frac{1}{2}\left(x_{n+1} + \frac{a}{x_{n+1}}\right) < x_{n+1} \equiv \left(x_{n+1} + \frac{a}{x_{n+1}}\right) < 2x_{n+1} \equiv \newline
	\frac{a}{x_{n+1}} < x_{n+1}$
\end{center}

Aqu\'i surgen tres posibilidades: 
\begin{enumerate}
	\item $x_{n+1} = 0$
	\item $x_{n+1} < 0$
	\item $x_{n+1} > 0$
\end{enumerate}

La primera posibilidad queda descartada ya que en la sucesi\'on de punto fijo aparece como denominador, y por lo tanto $x_n \neq 0$ para $n \geq 0$.

Suponiendo que vale la segunda posibilidad, entonces resulta que:

\begin{center}
	$\displaystyle{{\frac{a}{x_{n+1}} < x_{n+1}}} \equiv a > x_{n+1}^2 \equiv \sqrt{a} > |x_{n+1}| \equiv \sqrt{a} > -x_{n+1} \equiv -\sqrt{a} < x_{n+1} <$\footnote{Por hip\'otesis inductiva} $x_n < x_{n-1} < \dots < x_0 > \sqrt{a} \Rightarrow -\sqrt{a} < |x_{n+1}|
	$
\end{center}


