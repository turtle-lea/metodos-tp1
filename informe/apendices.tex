\subsection{Apéndice A: Enunciado}

\parindent = 0 pt
\parskip = 11 pt

El objetivo del trabajo pr\'actico consiste en implementar un programa que permita calcular, dado $\alpha \in
\mathbb{R}$, $1/\sqrt{\alpha}$. Para ello, se deber\'a considerar las funciones $f(x)$ y $e(x)$ definidas
anteriormente, distintos m\'etodos vistos en clase que permitan resolver el problema planteado y realizar un an\'alisis
completo del comportamiento de los mismos. 

Los requisitos m\'inimos a cumplir son los siguientes:

\begin{itemize}
\item Implementar el m\'etodo de Newton para la funci\'on $f(x)$. Incluir en el informe la demostraci\'on de
convergencia (Ejercicio 3, Pr\'actica 1). Para la funci\'on $e(x)$, implementar al menos dos m\'etodos (uno de los
cuales debe ser el de Newton).   
\item Para cada m\'etodo, estudiar experimentalmente la convergencia, tiempo de ejecuci\'on, cantidad de iteraciones,
criterios de parada, precisi\'on en el resultado, y cualquier otro par\'ametro que considere necesario evaluar. Realizar experimentos
computacionales considerando un rango amplio de valores posibles para $\alpha$ y distintos puntos iniciales
para los m\'etodos. Analizar y justificar detalladamente los resultados obtenidos.
\item Una vez fijados los mejores par\'ametros para cada m\'etodo, realizar una comparaci\'on entre las tres formas
alternativas de resolver el problema (Newton para $f(x)$, y Newton m\'as el otro m\'etodo para $e(x)$) en t\'erminos de
tiempo de ejecuci\'on, precisi\'on en la soluci\'on, cantidad de iteraciones, etc. Determinar experimentalmente que
variante seleccionar\'ia para su utilizaci\'on en la pr\'actica.
\end{itemize}

\subsection{Apéndice B: Código}

M\'etodo de Newton para la funci\'on $f(x)$:

\begin{verbatim}

	double Metodos::Newton_f_aux(int criterio,ostream& os,bool hacerBiseccion,
	                              int cantItBis,double x_0){
	double actual, anterior;
	if(hacerBiseccion){
		pair<double,double> seed =  functions.semillas_biseccion_f(cantItBis);
		actual = seed.first;
		anterior = seed.first;
	}
	else{
		actual = x_0;
		anterior = x_0;
	}
	int i = 0;
	os << actual << " " <<  functions.f(actual) << endl;
	while(!criterios.criterios(criterio,i,actual,anterior,functions.f(actual),
	      functions.f(anterior))){ 
		anterior = actual;
		actual = anterior - (functions.f(anterior)/functions.f_deriv(anterior));
		++i;
		os << actual << " " << functions.f(actual) << endl;
	}
	return actual;
}
\end{verbatim}

M\'etodo de la secante para $f(x)$:

\begin{verbatim}
	double Metodos::Secante_f_aux(int criterio,ostream& os,bool hacerBiseccion, int cantItBis,
	                              double seed1, double seed2){
	double actual, anterior;
	if(hacerBiseccion){
		pair<double,double> seed =  functions.semillas_biseccion_f(cantItBis);
		anterior = seed.first;
		actual = seed.second;
	}
	else{
		actual = seed1;
		anterior = seed2;
	}
	double prox;	//X_2
	int i = 0;
	while(!criterios.criterios(criterio,i,actual,anterior,functions.f(actual),
	           functions.f(anterior))){
		prox = actual - 
		  (functions.f(actual)*(actual-anterior)/
		    (functions.f(actual)-functions.f(anterior)));
		anterior = actual;
		actual = prox;
		++i;
		os << prox << " " << functions.f(prox) << endl;
	}
	return actual;
}
\end{verbatim}

M\'etodo $Regula-falsi$ para $f(x)$:

\begin{verbatim}
	double Metodos::Regula_falsi_f_aux(int criterio,ostream& os,bool hacerBiseccion,
	                                   int cantItBis,double seed1,double seed2){
	double positivo, negativo;
	if(hacerBiseccion){
		pair<double,double> seed =  functions.semillas_biseccion_f(cantItBis);
		positivo = seed.first;
		negativo = seed.second;
	}
	else{
		assert(functions.f(seed1)*functions.f(seed2) < 0);
		positivo = seed1;
		negativo = seed2;
	}
	double prox;
	int i = 0;

	while(!criterios.criterios(criterio,i,negativo,positivo,functions.f(negativo),
	             functions.f(positivo))){
		prox = negativo - 
		 (functions.f(negativo)*(negativo-positivo)/
		   (functions.f(negativo)-functions.f(positivo)));
		
		if(prox > 0) positivo = prox;
		else negativo = prox;
		++i;
		os << prox << " " << functions.f(prox) << endl;
	}
	return prox;
}
\end{verbatim}

No se incluyen los c\'odigos para la funci\'on $e(x)$ ya que son exactamente id\'enticos, cambiando $f(x)$ y $f_deriv(x)$ por $e(x)$ y $e_deriv(x)$ respectivamente.


\subsection{Apéndice C: Demostración}

Sea $f(x) = x^2 -a$ una funci\'on a la cual se le quiere encontrar raices.Para ello se propone una sucesi\'on de punto fijo $x_{n+1} = g(x_n) = \frac{1}{2}\left(x_n+ \frac{a}{x_n}\right)$. Veamos si efectivamente $g(x)$ corresponde con el m\'etodo de Newton.

Sea $f(x) = x^2 - a$ $\Rightarrow$ $f'(x) = 2x$. Por lo tanto, la sucesi\'on del m\'etodo de Newton es la siguiente: $g_{2}(x_n) = x_n - \frac{(x_{n}^2 -a)}{2x_n}$. 

Queremos ver que $g_2(x_n)=g(x_n)$ 

$\displaystyle g_{2}(x_n) = \displaystyle x_n - \frac{(x_{n}^2 -a)}{2x_n}$ $=$ $\displaystyle \frac{1}{2}\left(2x_n - \frac{x_{n}^2 - a}{x_n}\right)$ $=$ $\displaystyle \frac{1}{2}\left(\frac{2x_{n}^2 -x_{n}^2 + a}{x_n}\right) = \frac{1}{2}\left(\frac{x_{n}^2 + a}{x_n}\right) = \frac{1}{2}\left(x_{n} + \frac{a}{x_n}\right) = g(x_n)$

Luego hay que probar que si $x_0 > \sqrt{a}$ entonces $x_{n+1} < x_n$ para $n \geq 0$.

Ant\'es que nada probemos una propiedad que vamos a utilizar m\'as adelante:

\hspace{6.5cm}$x_0 > \sqrt{a} \Rightarrow x_n > \sqrt{a}$ (P1). 

Mediante inducci\'on, definimos el esquema inductivo a probar como:

Sea $P_1(n):=$ $x_0 > \sqrt{a} \Rightarrow x_n > \sqrt{a} $
\begin{itemize}
	\item Caso base: $P_1(0) \land P_1(1)$ 
	\item Caso inductivo: $P_1(n) \Rightarrow P_1(n+1)$
\end{itemize}

Probar $P_1(0)$ es trivial porque es exactamente el lado izquierdo de la implicaci\'on. Veamos $P_1(1)$

\hspace{4cm}$\displaystyle  x_1 > \sqrt{a} \equiv g(x_0) > \sqrt{a} \equiv \frac{1}{2}\left(x_0+ \frac{a}{x_0}\right) > \sqrt{a} \equiv x_0+ \frac{a}{x_0} > 2\sqrt{a}\equiv$


\hspace{4cm}$\displaystyle  x_{0}^2+ a > 2x_{0}\sqrt{a}$ \footnotemark[1] $\equiv$	$x_{0}^2 - 2x_{0}\sqrt{a} + a > 0$ $\equiv$ $\left(x_{0} - \sqrt{a}\right)^2 > 0$

1: Como $x_0 > \sqrt{a} > 0$ entonces la desigualdad no cambia de sentido.

Como $x_0 > \sqrt{a}$ entonces vale que $\left(x_{0} - \sqrt{a}\right)^2 > 0$.

Luego, para el caso inductivo se procede de la misma forma que en el caso base, pero intercambiando $x_1$ por $x_{n+1}$ y $x_0$ por $x_n$, de modo que queda:

\hspace{4cm}$\displaystyle  x_{n+1} > \sqrt{a} \equiv g(x_n) > \sqrt{a} \equiv \frac{1}{2}\left(x_n+ \frac{a}{x_n}\right) > \sqrt{a} \equiv x_n+ \frac{a}{x_n} > 2\sqrt{a}\equiv$


\hspace{4cm}$\displaystyle  x_{n}^2+ a > 2x_{n}\sqrt{a}$ \footnotemark[2] $\equiv$	$x_{n}^2 - 2x_{n}\sqrt{a} + a > 0$ $\equiv$ $\left(x_{n} - \sqrt{a}\right)^2 > 0$


2: Como $x_n > \sqrt{a} > 0$ entonces la desigualdad no cambia de sentido.

Y nuevamente, como $x_n > \sqrt{a}$, entonces vale que $\left(x_{n} - \sqrt{a}\right)^2 > 0$.

Lo interesante de la propiedad (P1) es que nos permite implicar que $(\forall n \geq 0) x_n > 0$.

Ahora, volviendo al problema original.

Sea $P(n):=$ $x_0 > \sqrt{a} \Rightarrow x_{n+1} < x_{n}$
\begin{itemize}
	\item Caso base: $P(0) = x_0 > \sqrt{a} \Rightarrow x_1 < x_0$
	\item Caso inductivo: $((\forall n\geq 0)$ $P(n)) \Rightarrow P(n+1)$
\end{itemize}

Caso base:


\hspace{4cm}$\displaystyle x_1 < x_0 \equiv g(x_0) < x_0 \equiv \frac{1}{2}\left(x_0 + \frac{a}{x_0}\right) < x_0 \equiv \left(x_0 + \frac{a}{x_0}\right) < 2x_0 \equiv$


\hspace{4.5cm}$\displaystyle \frac{a}{x_0} < x_0 \equiv a < x_{0}^2$ $\Rightarrow \sqrt{a} < |x_0| \equiv \sqrt{a} < x_0 \equiv P(0)$

Caso inductivo:
 
\hspace{4cm}$\displaystyle  x_{n+2} < x_{n+1} \equiv g(x_{n+1}) < x_{n+1} \equiv \frac{1}{2}\left(x_{n+1} + \frac{a}{x_{n+1}}\right) < x_{n+1} \equiv $


\hspace{4cm}$\displaystyle  \left(x_{n+1} + \frac{a}{x_{n+1}}\right) < 2x_{n+1} \equiv \frac{a}{x_{n+1}} < x_{n+1} \equiv a < x_{n+1}^2 \equiv $


\hspace{4cm}$\displaystyle  \sqrt{a} < |x_{n+1}| \equiv \sqrt{a} < x_{n+1} <\footnotemark[3]$  $x_n < \ldots < x_0 \Rightarrow \sqrt{a} < x_0$


3: Por hip\'otesis inductiva.